\chapter{Summary (Danish)}
\begin{otherlanguage}{danish}

Formålet med denne projekt er at præsentere en metode, der påviser eksistensen af “physical colocation” på baggrund af Bluetooth RSSI-målinger. Der indsamles to typer data ved hjælp af mobiltelefoner, som derefter samles i ét datasæt. Det endelige datasæt bruges til at analysere tre algoritmer: kunstige “neural” netværk, logistisk regression og Naive Bayes’. Algoritmernes parametre og de anvendte “features”  varieres og resultaterne sammenlignes. Det bruges til at konkludere hvor godt eksistensen af “co-location” underbygges af de indsamlede data såvel som hvor dyre algoritmerne er i køretid. Artiklen afsluttes med at anbefale dén kombination af algoritme og “feature”, der gav de bedste resultater

\end{otherlanguage}